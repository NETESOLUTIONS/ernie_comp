%%%%%%%%%%%%%%%%%%%%%%% file template.tex %%%%%%%%%%%%%%%%%%%%%%%%%
%
% This is a general template file for the LaTeX package SVJour3
% for Springer journals.          Springer Heidelberg 2010/09/16
%
% Copy it to a new file with a new name and use it as the basis
% for your article. Delete % signs as needed.
%
% This template includes a few options for different layouts and
% content for various journals. Please consult a previous issue of
% your journal as needed.
%
%%%%%%%%%%%%%%%%%%%%%%%%%%%%%%%%%%%%%%%%%%%%%%%%%%%%%%%%%%%%%%%%%%%
%
% First comes an example EPS file -- just ignore it and
% proceed on the \documentclass line
% your LaTeX will extract the file if required
\begin{filecontents*}{example.eps}
%!PS-Adobe-3.0 EPSF-3.0
%%BoundingBox: 19 19 221 221
%%CreationDate: Mon Sep 29 1997
%%Creator: programmed by hand (JK)
%%EndComments
gsave
newpath
  20 20 moveto
  20 220 lineto
  220 220 lineto
  220 20 lineto
closepath
2 setlinewidth
gsave
  .4 setgray fill
grestore
stroke
grestorew
\end{filecontents*}
%
\RequirePackage{fix-cm}
%
%\documentclass{svjour3}                     % onecolumn (standard format)
%\documentclass[smallcondensed]{svjour3}     % onecolumn (ditto)
\documentclass[smallextended]{svjour3}       % onecolumn (second format)
%\documentclass[twocolumn]{svjour3}          % twocolumn
%
\smartqed  % flush right qed marks, e.g. at end of proof
%
\usepackage{graphicx}
\usepackage{xcolor}
\usepackage{listings}
%
% \usepackage{mathptmx}      % use Times fonts if available on your TeX system
%
% insert here the call for the packages your document requires
%\usepackage{latexsym}
% etc.
%
% please place your own definitions here and don't use \def but
% \newcommand{}{}
%
% Insert the name of "your journal" with
% \journalname{myjournal}
%
\begin{document}

\title{A Vignetted View of Computer Science through Citation Analysis}
%\thanks{Grants or other notes
%about the article that should go on the front page should be
%placed here. General acknowledgments should be placed at the end of the article.}
\subtitle{Salton and Bergmark Redux}
%\titlerunning{Short form of title}        % if too long for running head

\author{Sitaram Devarakonda  \and
	Dmitriy Korobskiy \and
        Tandy Warnow \and
        George Chacko }

%\authorrunning{Short form of author list} % if too long for running head

\institute{Sitaram Devarakonda,  \at
              Netelabs, NET ESolutions Corporation, McLean, VA \\
              %Tel.: +123-45-678910\\
              %Fax: +123-45-678910\\
              \email{sitaram@nete.com}           %  \\
%             \emph{Present address:} of F. Author  %  if needed
           \and
              Dmitriy Korobskiy \at
              Netelabs, NET ESolutions Corporation, McLean, VA  \\
              \email{dk@nete.com}
           \and
           Tandy Warnow \at
              Dept of Computer Science, University of Illinois Urbana-Champaign, Champaign IL \\
              \email{warnow@illinois.edu}
           \and
           George Chacko \at
              Netelabs, NET ESolutions Corporation, McLean, VA \\
              \email{netelabs@nwete.com}
}

\date{Received: date / Accepted: date}
% The correct dates will be entered by the editor

\maketitle

\begin{abstract}
Computer science has experienced dramatic growth and diversification over the last twenty years. Towards a current understanding of the structure of this discipline, we analyze a cohort of the computer science literature.  For insight on the features of this cohort and the relationship within its components, we construct article level clusters based on either direct citations or co-citations, and reconcile them with major and minor subject categories in the All Science Journal Classification (Scopus). [What do we find?].

\begin{itemize}
\item understanding structure of communications within a field and with adjacent fields
\item extending Salton and Bergmark and staying high-level
\item combining article level and journal level 
\item two clustering methods
\item Graclus- all inclusive, unweighted edges
\item Co-citations, highly selective and weighted edges
\item intersection of Graclus and co-citation
\item close out
\end{itemize}

\keywords{Bibliometrics \and Clustering \and Research Evaluation \and Computer Science}
% \PACS{PACS code1 \and PACS code2 \and more}
\subclass{01A85 \and 01A90} %\and more}

\end{abstract}

\section{Introduction}
\label{intro}

Computer science and its applications have experienced rapid growth and diversification over the last twenty years. As observed in a 2017 US National Academies Report, ``A wide range of jobs in virtually all sectors demand computing skills to an unprecedented extent. And every academic discipline finds itself incorporating computing into its research and educational mission"~\cite{nas_2017}. More recently, the collective influence of the Internet of Things (IoT), `big' data, accessible cloud computing, and advances in artificial intelligence have been postulated as a recent driver for growth and evolution~\cite{siebel2019_digital}. Given this powerful influence some understanding of the present state and structure of computer science and its relationship to other fields could inform planning and policy making at multiple levels from national level funding all the way down to faculty hiring strategy. For the same reason of broad influence, computer science can also serve as a model system for studying interdisciplinarity. 

Noting that that the scientific literature serves a rich source of information to study the structure and historical development of a field, 
Salton and Bergmark conducted a historical study  in 1979 of the computer science literature (419 computer articles published in 1974,  and 3,812 references cited in these articles)~\cite{salton_citation_1979}. These authors described the global structure of computer science as comprising three main areas: (i) theoretical foundations, such as theory of computation, (ii) hardware and computer systems, such as architecture,  and (iii) software, such as programming systems.  Related areas noted were (a) mathematics of computing, such as numerical analysis, (b) special software topics, such as operating systems, (c) data management and database systems, (d) methodologies valid for multiple applications, such as algebraic manipulation, (e) computer applications, such as computer graphics, and  (f) non-technical aspects, such as computer education. 

Looking beyond Salton and Bergmark's historical triad of theoretical foundations, hardware and computer systems, and software, the Computing Classification System (CCS) published by the Association for Computing Machinery now consists of 13 top-level areas that reflect a more current view of the field~\cite{acm_ref}. This classification also addresses relationships with other fields under the category of Applied Computing. However, an easy way to map scientific publications to the CCS, especially interdisciplinary articles or those from proximal fields, does not seem to be available. 

Other classification systems are available, such as the All Science Journal Classification (ASJC) developed and maintained by Scopus, the Web of Science research and categorical classifications from Clarivate Analytics, and the NSF classification system~\cite{nsf_classification} [add citations for Scopus and Web of Science]. Scopus and the Web of Science also include comprehensive bibliographies with citation links and proprietary unique identifiers for publications. 
All three systems rely on applying one or more journal-derived labels to articles.  
A logical prediction, given the diversity of articles within journals, is limited specificity at the article level. 
Shu and colleagues recently noted in a comparative study of the Chinese Science Citation Database (CSCD)  and the Web of Science that 46\% of articles did not belong to the discipline of the journal they were published in~\cite{shu_comparing_2019}. 
Others have also discussed and critiqued disciplinary assignments using journal-based classifications~\cite{wang_large-scale_2016,perianes-rodriguez_comparison_2017}. 
Article classification systems have been constructed that escape some of the criticisms of journal-based classification~\cite{traag_louvain_2019,boyack_classification_2014,waltman_new_2012}, but do not enjoy widespread use. [Add reference to Scandinavian paper in QSS].
 
In this study, we extend the work of Salton and Bergmark by combining article and journal approaches to study trends in computer science, while also considering connections to other fields, especially biology, which dominates the total number of scientific publications and is citationally active [need to tweak]. As a source of computer science literature, we used DBLP, a reference bibliography for computer science~\cite{dblp_ref}. The DBLP bibliography covers publications from computer science and includes publications from hybrid fields, where they are considered pertinent to computer science research. We describe below the high-level interactions of computer science within the Physical Sciences, and with the Social Sciences, Life Sciences, and Health Sciences [What did we find? Some kind of preview sentence!].

\section{Materials and Methods}
\label{sec:methods}


 \subsection{Overview}
 In our study of the structure of a twenty year cohort of the computer science literature, we chose to use both traditional journal-based and article-based approaches. Considering that traditional disciplines `may only partly reflect the actual organization of today's scientific research'~\cite{waltman_new_2012}, we constructed article clusters at high levels of aggregation using citations to examine the  computer science literature (as represented in our \emph{comp} dataset, described below) and mapped these clusters to journal-based categories.
 
 Clustering of publications is commonly accomplished through direct citation, bibliographic coupling, and co-citation, with direct citation being proposed as as the best approach to concentrate citation links~\cite{klavans_which_2017}. Accordingly, we used direct citation links as the basis for cluster formation, and also co-citation to obtain an alternative view.

For the purpose of this study, our working definition of the computer science literature was all publications in the DBLP bibliography that (i) had a digital object identifier (DOI), and (ii) could also be matched to article identifiers in the Scopus bibliography. The underlying assumption is that records in DBLP are greatly enriched for computer science even while recognizing that coverage may be incomplete and possibly biased. Cross-matching DBLP publications to records in the Scopus abstract and citation database of peer-reviewed literature, enabled us to harvest the richer links in Scopus as well as extract links to publications from other disciplines. This cross-matching to Scopus also allows the use of both journal and article based classifications when clustering documents. Using both direct citations and co-citations as the basis for clustering, we used DBLP articles from journals and conference proceedings. We reconciled these clusters with the All Science Journal Classification (ASJC) developed and maintained by Scopus through a combination of automated and manual procedures, producing a dataset of  2,685,356 publications.

\subsection{Data}
A stable release of the DBLP computer science bibliography~\cite{dblp_ref} consisting of 7,079,994 records was downloaded as dblp-2018-08-01.xml.gz. Slightly over 95\% of the publications within were published after 1996. Publications were parsed from the XML source file and loaded into a PostgreSQL database. As part of implementing a larger data platform for research evaluation~\cite{GithubERNIE2019}, we have previously parsed the Scopus dataset, presently at over 87.8 million publications, into a custom schema in a PostgreSQL database. The total number of publications in Scopus labeled with major subject area of Computer Science (in turn a subset of the Physical Sciences subject area) is 5,835,160. Records in the DBLP dataset were matched to Scopus identifiers using digital object identifiers (DOIs). This procedure resulted in a dataset of 2,685,356 DBLP publications with Scopus identifiers where 1,278,322 (47.6\%) were labeled as article and 1,407,034 (52.4\%) as conference proceedings.  References cited by these publications were then extracted from Scopus (7,129,006 records), resulting in a total of 8,000,411 
publications and references. 
We represented these data using a graph where the 8,000,411 nodes represent
publications and references  and the 44,296,381 undirected edges represent citations within the dataset. 
We refer to these data as the  \emph{comp} dataset (Fig~\ref{fig:ar_cp_annotation}, Table~\ref{tab:comp}).  
 
\subsection{Clustering methods} In applying both clustering by direct citation and by co-citation, we attempted to consider, wherever possible, the criteria articulated by \v{S}ubelj, van Eck, and Waltman~\cite{subelj_clustering_2016} that (i) the largest cluster should be no more than 10 times the smallest one, (ii) small clusters should be eliminated, (iii) small changes and replicates should yield similar results (``stability"),  (iv) computing time should be minimized where possible, and (v) the clustering should seem reasonable on a qualitative level (``intuitive sensibility").

\paragraph{Clustering by Direct Citation.} Graclus~\cite{graclus_2007} is a spectral graph clustering package that optimizes  various clustering criteria, including normalized cut, ratio cut, and ratio association, that has previously been applied to citation data~\cite{subelj_clustering_2016}. 
We used v1.2 in our experiments. The \emph{comp} dataset was formatted as an undirected graph, stored in a file with a header line indicating the number of nodes and edges, and used as input to Graclus, which requires the number of clusters to be formed as an input parameter. In preliminary experiments, we varied the number of clusters to be formed between 10 and 50 clusters (data not shown).  At around 20 clusters, clusters size was relatively stable with the largest cluster containing roughly 10 times the number of nodes in the smallest one~\cite{subelj_clustering_2016}. Consequently, we used Graclus to generate 20 clusters, labeled 0-19 (Table~\ref{tab:graclus}), analogous to Level 1 of Waltman and van Eck's mapping of nearly 10 million publications~\cite{waltman_new_2012} but focused on the DBLP bibliography.

We also used conductance, as defined in Shun et al.~\cite{shun_parallel_2016}, to  evaluate clustering by direct citation (smaller is better), noting that conductance has been found to be a good metric for this purpose~\cite{emmons2016analysis,almeida_2012}. 
%Tandy, George - add reference to Zaki paper as well
In our analysis, we saw that the last cluster (cluster 19) had a much larger conductance value than the other clusters and also had the smallest
number of nodes.
We then examined
results obtained using Graclus with two other numbers of clusters (18 and 22), and
in each case, the highest numbered cluster  had the greatest conductance value and also the smallest number of nodes.
These results suggest that Graclus produces a final cluster that effectively serves as a container for `left over publications' during the clustering procedure. 
Therefore, we limited our consideration of cluster 19 (the final of the twenty clusters)  when interpreting results.  The remaining 19 clusters had conductance values ranging from 0.09 to 0.25 with a median conductance of 0.15 (Fig.~\ref{fig:graclus_comparison_conductance}, Table~\ref{tab:comp}).

\paragraph{Clustering by Co-Citation.}

For an alternate view of these DBLP data, we constructed clusters using co-citation, the frequency with which a pair of articles is cited by other articles \cite{small_co-citation_1973,marshakova-shaikevich_co-citation_1973}. % and has been used for clustering publications. %George, add citation 
Co-citation, first described independently by Small and Marshakova in 1973 \cite{marshakova-shaikevich_co-citation_1973}, provides insight into the emergence of new ideas derived from the association of previously independent ones. Unlike clustering by direct citation, where every input publication is assigned to a cluster and every citation is weighted equally, the co-citation relationship between papers is weighted to represent the strength of the co-citation history.
Because this produces a weighted graph, clustering methods that address weights are required.
Furthermore, co-citation produces relatively weak inter-cluster interactions, requiring additional modifications to standard clustering approaches \cite{boyack_cocitation_2010,boyack_improving_2013,small_structure_1974,small_clustering_1985}. 

%For clustering, we applied a variant of variable level clustering (VLC, described below) followed by 600 iterations of agglomeration of the clusters resultant from VLC since further rounds of agglomeration resulted in the development of a single large cluster (Fig.~\ref{fig:merge_select}). To focus on major clusters, we also restricted agglomeration to clusters containing at least 10 nodes.  Consequently, the 20 largest clusters were used for further analysis and comparison with clusters generated by direct citation. The largest of these 20 clusters contained 1,276 nodes and the smallest contained 136, with total nodes summing to 8,042.

We used a modification of variable level clustering combined with agglomerative clustering, an approach developed in 1985 by Small and Sweeney~\cite{small_clustering_1985} for co-citation analysis.
Variable level clustering involves applying a threshold (below which all edges in a graph are deleted) then iteratively selecting edges with the highest normalized co-citation value and extracting connected components from the graph as clusters for each edge in turn. Three parameters are needed: (i) a threshold or starting level based on a quantile of normalized co-citation frequency, (ii) a level increment, and (iii) a maximum cluster size. An issue is the generation of very large clusters by chaining via low edge weights.  Thus, at each iteration, any cluster exceeding the maximum cluster size  is returned to the process and a higher threshold is applied to break such clusters.

In our implementation of variable level clustering (Fig.~\ref{fig:quad-chart}(a)), we first calculated the number of citations accumulated across Scopus for all 2,685,356 articles in \emph{comp}. After discarding those publications without any citations, we restricted further analysis to highly cited articles--those in the 90th percentile of citations or higher--resulting in a dataset of size 212,311. We then identified 4,318,305 publications in Scopus that cite these 212,311 articles from \emph{comp}. For each of the 4,318,305 citing publications in turn, all possible ${n \choose 2}$ reference pairs were generated from a publication's cited references, where $n$ is the number of references in a publication. The cited reference pairs this generated were then restricted to those in the set of 212,311 highly cited papers previously identified. A total of 46,463,117 unique co-cited pairs were thus obtained. 
The frequency of these co-cited pairs was then computed across the \emph{comp} dataset  and normalized using Salton's cosine formula~\cite{salton_citation_1979}. These data were represented in a graph where each node was a publication and the weighted edge between the pair was the normalized co-citation frequency. \par 

We applied variable clustering with initial parameters as follows: the threshold  $t$  is initially set to  the median normalized co-citation frequency (quantile=0.5), increment $i = 0.1$, and maximum cluster size, $mcs=200$. Thus, at start, all edges below the median normalized co-citation frequency were deleted. Clusters were formed by assembling connected components from each co-cited pair beginning with the heaviest edge weight. Clusters below size 100 were retained and any cluster larger than 200 nodes was carried over to the next round. The threshold, $t$, was then incremented by 0.1 and the process repeated while progressively incrementing $t$.  We used a bi-phasic approach where in which $t$ ranged from 0.5--0.9, after which $i$ was reduced to 0.01 for the range $0.9 \leq t \leq 0.99$. A final threshold of $t$=0.999 was applied to break the single remaining large cluster.  Using this approach, 22,232 clusters containing 84,591 nodes were generated, each containing less than 100 nodes. Clusters containing only 2 nodes were discarded, bringing the total number of clusters down to 10,298. The publications in these 10,298 clusters were overwhelmingly drawn from the Physical Sciences (one of the four top level categories in the Scopus ASJC classification), of which computer science is a sub-category (Fig.~\ref{fig:quad-chart}).

Agglomerative clustering was then performed on these 10,298 clusters to generate higher-order clusters. To focus on larger clusters, only those with at least 10 nodes were used as input. Briefly, each cluster was now treated as a node and the edge weight between two clusters was assigned to the maximum edge weight of all edges between the nodes in the two clusters. Edges were arranged in descending order. The first pair of clusters was merged and its edge weight with other interacting clusters was recalculated, again based on maximum edge weight. All edges were then re-ordered as before and the next pair of clusters was merged. The process was halted after 600 rounds to prevent large outlier clusters being generated (Fig.~\ref{fig:quad-chart}).

\section*{Results}
\label{sec:results}


\subsection{Data}
The process of selection and matching resulted in a dataset of  publications (Materials and Methods). Of 4,291,130 DBLP publications with DOIs, only  2,685,356 had corresponding DOIs in Scopus. Of these roughly 2.68 million publications, approximately 2.07 million were assigned ASJC codes in Scopus corresponding to Computer Science (major subject area with 13 minor subject areas (Table~\ref{tab:comp}), with the balance of 610,000 publications non-exclusively shared between 26 different major subject areas ranging from Engineering (330,048) to Dentistry (Fig.~\ref{fig:ar_cp_annotation}). The set of 2.07 million publications classified under the major subject area Computer Science in Scopus spanned all 13 minor subject areas with Software 30.3\% being the largest component and Computer Science (miscellaneous) 0.9\% the smallest. Publications in the \emph{comp} dataset labeled Theoretical Computer Science (409,082) are classified under the the major subject area of Mathematics rather than Computer Science (Table~\ref{tab:comp}). Overall, 38.86\% of the publications were assigned to one minor subject area, 78.1\% to two minor subject areas, 90.9\% to three minor subject areas, and 0.0003\% assigned to 8 different minor subject areas. 

%George, please consider revising this text
%the sets of computer science publications in DBLP and Scopus are partially overlapping rather than DBLP being a perfect subset of Scopus. [the point being that we should justify our use of both journal and article level classifications].

\subsection{Clustering by Direct Citation}  

We constructed article-level clusters of this computer science dataset at a sufficiently high level of aggregation to avoid cognitive challenge and cross-matched them to  the  Scopus ASJC classification. 
To focus on relatively high signal, we only considered Scopus ASJC minor subject area categories that accounted for at least 15\% of the publications in each cluster. 

Figure~\ref{fig:heatmap} permits examination of these data from two perspectives: (i) rows: the clusters that map to a given ASJC minor subject area (ii) columns: ASJC minor subject areas that comprised at least 15\% of the publication in a cluster. Under these conditions of clustering and a threshold of 15\%, 31 of the 334 ASJC minor subject areas are represented. Unsurprisingly, the broad categories Computer Science Applications, Software, and Electrical and Electronic Engineering register in 16, 12, and 10 clusters respectively, while Artificial Intelligence mapped to 7 different clusters. At the other end of the range, 12 of the 31 ASJC minor subject areas were each detected in a single cluster. 

From the alternate perspective (columns), Cluster 17 was the most diverse and contained publications annotated with 8 minor subject area labels: Biochemistry, Chemistry(all), Genetics, Molecular Biology,  Statistics \& Probability, Computational Theory \& Mathematics, and Computational Mathematics. Cluster 19 mapped to two areas but was excluded from qualitative analysis because of its high conductance value. Of the remaining clusters, Cluster 2 represents interactions between the minor subject areas Theoretical Computer Science, Discrete Mathematics and Combinatorics, and Applied Mathematics as well as the more generic Computer Science (all). Clusters 3 and 4 include Computer Networks and Communications, and Cluster 18 (Artificial Intelligence, Cognitive Neuroscience, and Neurology) and clusters 5--9 include Management Science, Operations Research, Information Systems, Modeling and Simulation, and Human-Computer Interaction.

These data suggest that areas central to computer science in 2019 (Salton and Bergmark's historical triad of hardware, software, and theory) are more likely to be found in multiple clusters than peripheral topics.
%, whereas areas such as biology, chemistry, and statistics are more restricted in distribution.  
A second inference is that, in some cases, journal-based classification and our article clusters align fairly well (Hardware and Architecture). 
A third inference is that the ASJC minor subject area ``Computer Applications" is relatively non-specific, and publications thus labeled are nearly ubiquitous (present in 16 out of 20 clusters).  Finally,   for the DBLP dataset, interactions with fields outside computer science (e.g., Biochemistry)  are  typically restricted to a single cluster or found in a small number of clusters.

We also note that these clusters cannot be adequately characterized by using a single CCS classification, a challenge that has been previously noted \cite{waltman_new_2012}.
For example, we
manually matched the top 25--50 most heavily cited publications in each cluster to corresponding categories in the CCS. This was feasible in some cases, but not in all. For example, clusters 0 and 1 mapped reasonably well to the top level categories Hardware and Computer Systems Organization,  but 
the top cited papers often derived from biology and yet biology was clearly not  representative of any of these affected clusters.


\subsection{Clustering by Co-citation.} 
For an alternate examination of the data, we used co-citation frequencies to cluster the DBLP dataset as described above. Figure \ref{fig:heatmap_cocit} shows a heatmap in which clusters constructed by co-citation are mapped to Scopus ASJC minor subject areas, with the top subfigure showing results for those labels that
account for at least 15\% of the publications in a cluster, and the bottom subfigure showing results for those labels that account for at least 10\% of the publications.
%Mappings are only shown if they account for at least 15\% of the publications in a cluster (upper panel). 
Under these conditions, only 17 of 20 clusters mapped to minor subject areas where at least 15\% of the nodes in a cluster were labeled with a given subject area. When the threshold was reduced to 10\%, all 20 clusters indicated at least one mapping to a minor subject area. We interpret these data as indicative of two kinds of co-citational linkages (i) intra-cluster (ii) cross cluster. The latter predominates and is detected at lower thresholds (lower panel). No cluster maps to more than three minor subject area at the 15\% threshold,  in contrast to a maximum of eight minor subject areas for clustering by direct citation (Fig. 3, Cluster 17). It should be noted, however, that the size of clusters constructed by co-citation is much smaller than those generated by direct citation through Graclus. 
%George, add some comments and discussion

We also mapped the contents of clusters generated by direct citation and co-citation to each other, Fig~\ref{fig:graclus_cocit_fig}.

\section{Discussion} In this article, we revisited a characterization of computer science from 1979~\cite{salton_citation_1979} in the light of expansion and diversification of this field. We analyzed considerably more data, 2.68 million publications versus 391, and leveraged access to two bibliographic databases, DBLP and Scopus. The former identifies a sample of the scientific literature that is computer science and the latter (roughly 10 times larger at 87.XX million publications) enabled analyzing citations across the four major groups of Physical Science, Social Sciences, Life Sciences, and Health Sciences in the Scopus ASJC classification.

Obvious critiques of our approach are: (i) use of the DBLP dataset, which may not capture all aspects of computer science and its interactions with other fields, (ii) partitioning DBLP into a weakly justified number of clusters and arbitrarily defined thresholds for detection, and (iii) mapping the results of this clustering against a classification designed around journals rather than individual articles. 
Our focus on high-level features and preference for simplicity and intuitiveness in our choice of methods may mitigate these concerns. 

Reconciling direct citation clusters to the Scopus ASJC classification yielded results that are consistent with the observation of Waltman and van Eck (2012)~\cite{waltman_new_2012}, `that traditional disciplines such as those just mentioned only partly reflect the actual organization of today’s scientific research'. Of interest to us, was the single obviously multidisciplinary cluster in which at least 15\% of its component publications were labeled with the ASJC minor subject areas Biochemistry, Chemistry, Computational Mathematics, Computational Theory and Mathematics, Computer Science Applications, Genetics, Molecular Biology, Statistics \& Probability, followed by the cluster mapping to Artificial Intelligence, Cognitive Neuroscience, and Neurology.  Perhaps at the opposite end of this spectrum is the more insular cluster that maps to Hardware and Architecture, Electrical and Electronic Engineering, and Software. To us, these data suggest that some subfields within computer science are primarily inward looking (remain concerned with fundamental questions in computer science), while others are more actively engaged with fields external to computer science.   

Analyzing clusters generated using co-citation links is more challenging. The data suggest two patterns that may co-exist. One in which the nodes in a co-citation cluster are detected at the threshold applied mostly from a single or two direct citation clusters. The second, which may account for more than 50\% of the nodes in a co-citation cluster, in which these nodes are distributed across all 20 direct citation clusters. The latter is intriguing in its implications of a web of such citational links gluing the body of scientific literature together.

Emerging interactions of computer science and other disciplines, so papers are related by the communities they impact (co-citation clustering). [Over to Tandy]
 
\emph{need to cite Granovetter somewhere for definition of strong and weak ties that Small refers to}
\clearpage

%%%%%%%%%
\section*{Figures}
%%%%%%%%%

\begin{figure}[ht]
% Use the relevant command to insert your figure file.
% For example, with the graphicx package use
  \includegraphics[scale=0.6]{ar_cp_ratio.pdf}
% figure caption is below the figure
\caption{Summary of DBLP data cross-matched with Scopus. 2,685,356 publications from DBLP were cross-matched with Scopus and then grouped by the 27 major subject areas in the ASJC (Scopus) classification. The largest number of publications are contributed by Computer Science; Engineering; Mathematics; and then by Social Sciences; Decision Sciences; Physics and Astronomy; Medicine; and Biochemistry, Genetics, and Molecular Biology. Publications were further annotated with respect to being either articles \emph{(ar)} or conference proceedings \emph{(cp)}. For this dataset, the major subject area of Computer Science with 1,194,623(cp) \& 879,396 contributed the most publications while Dentistry with 0 (cp) \& 1 (ar)) contributed the least.}
\label{fig:ar_cp_annotation}       % Give a unique label
\end{figure}

\newpage

\begin{figure}[ht]
\centering
% Use the relevant command to insert your figure file.
% For example, with the graphicx package use
  \includegraphics[scale=0.4]{graclus_comparison.pdf}
% figure caption is below the figure
\caption{Conductance Measurements of Clusters Generated by Graclus of the direct citation dataset.  2,685,356 DBLP publications, 7,129,006 cited references, and 44,296,381 citations were clustered using Graclus into 18 (grac\_18), 20 (grac\_20), or 22 (grac-22) clusters. Conductance, $\phi(S)$, was measured for these clusters considering only the edges between publications using the formula: $\phi(S)=|\partial(S)|/min(vol(S),2m-vol(S)$, where $\partial(S)$ is the boundary (number of edges leaving a set), vol(S) is volume of a set of vertices as the sum of the degrees of the vertices in a set, and $m$ is the number of undirected edges in a set~\cite{shun_parallel_2016}.}
\label{fig:graclus_comparison_conductance}       
\end{figure}

\newpage

\begin{figure}[ht]
% Use the relevant command to insert your figure file.
% For example, with the graphicx package use
  \includegraphics[scale=0.45]{scopus_dblp_graclus3.pdf}
% figure caption is below the figure
\caption{Heat map for the clustering obtained by direct citation. The  y-axis (rows) correspond to topics, defined by Scopus characterizations, and the
x-axis (columns) represent the 20 different clusters.  
Each cluster is characterized by topics that label at least 15\% of the publications in the cluster.
 }
\label{fig:heatmap}       % Give a unique label
\end{figure}

\begin{figure}[ht]
\centering
\includegraphics[width=0.9\textwidth]{quad-chart-narrow.pdf}
\caption{Co-citation analysis.
(a) Schematic representation of variable clustering protocol modified from Small and Sweeney (1985) \cite{small_clustering_1985}. Three parameters are specified: (i) a threshold or starting level based on a quantile of normalized co-citation frequency, (ii) a level increment, and  (iii) a maximum cluster size. Input data is a set of co-cited publications with edge-weight defined by normalized co-citation frequencies. Green clusters are within the max cluster size. At the initial threshold, $t1$, a single cluster below the maximum cluster size, $mcs$ (green), along with one large cluster above it (red) are generated. As the threshold is incremented to $t2$, additional clusters of acceptable size is generated. The cascade continues to completion, which is defined by all clusters being of size less than or equal to the $mcs$. In this schematic, five rounds are adequate for the process to run to completion.
(b) The distribution of publications (using fractional counting) across four top-level ASJC subject areas after applying variable level clustering as in a)
(c) The Venn diagram of the fractional counting given in (b).
(d) The distribution of cluster sizes (logarithmic y-axis) as a function of the number of iterations of the agglomerative clustering technique; note that the largest cluster is extremely large when the number of iterations exceeds 600.
}
\label{fig:quad-chart}
\end{figure}



\newpage

\begin{figure}[ht]
% Use the relevant command to insert your figure file.
% For example, with the graphicx package use
  \includegraphics[scale=0.45]{scopus_dblp_heatmap3.pdf}
% figure caption is below the figure
\caption{Heat map for the clustering obtained by co-citation, using two thresholds for inclusion (top: 15\%, bottom: 10\%). The  y-axis (rows) correspond to topics, defined by Scopus characterizations, and the
x-axis (columns) represent the 20 different clusters.  
Each cluster is characterized by topics that label at least the required minimum percentage of publications in the cluster (top: 15\%, bottom: 10\%).   }
\label{fig:heatmap_cocit}       % Give a unique label
\end{figure}
\newpage

\begin{figure}[ht]
% Use the relevant command to insert your figure file.
% For example, with the graphicx package use
  \includegraphics[scale=0.45]{graclus_cocit_fig.pdf}
% figure caption is below the figure
\caption{Intersections between clusters generated using direct citation and co-citation features. x-axis: Clusters generated by Graclus number 0-19. y-axis: Cluster generated by variable level clustering and agglomerative clustering using a modification of Small and Sweeney (1985)~\cite{small_clustering_1985}. Point size (perc) is the percentage of a co-cited cluster that maps to a corresponding Graclus cluster. A minimum threshold of 15\% was set. Graclus cluster 19, the 20th cluster, did not map to any cluster on the y:axis.}
\label{fig:graclus_cocit_fig}       % Give a unique label
\end{figure}
\newpage
\clearpage
\section*{Tables}

\begin{table}[ht]
\caption{\emph{denominator is 2090947 (some source ids got lost via inner joins and code restrictions) Statistics regarding the 20 clusters produced using direct citation. The \emph{comp} dataset 
2685021 is because of inner joins on ASJC Explain conductance calculations~\cite{shun_parallel_2016}}}
\label{tab:comp}       
\begin{tabular}{lrccc}
\hline\noalign{\smallskip}
minor\_subject\_area & Percent of Publications \\
\noalign{\smallskip}\hline\noalign{\smallskip}
Software & 30.1 \\
Computer Science Applications & 25.0 \\ 
Computer Networks and Communications & 21.5 \\
Theoretical Computer Science & 19.6 \\ 
Computer Science(all) & 19.4 \\ 
Artificial Intelligence & 14.2 \\ 
Information Systems & 11.8 \\ 
Hardware and Architecture & 11.8 \\ 
Signal Processing & 9.8 \\ 
Computer Vision and Pattern Recognition & 9.4 \\ 
Computational Theory and Mathematics & 7.9 \\ 
Human-Computer Interaction & 7.4 \\ 
Computer Graphics and Computer-Aided Design & 5.2 \\ 
Computer Science (miscellaneous) & 0.9 \\ 
\noalign{\smallskip}\hline
\end{tabular}
\end{table}
\newpage


\begin{table}[ht]
\caption{\emph{Statistics regarding the 20 clusters produced using direct citation. The \emph{comp} dataset 
2685021 is because of inner joins on ASJC Explain conductance calculations~\cite{shun_parallel_2016}}}
\label{tab:graclus}       
\begin{tabular}{lrccc}
\hline\noalign{\smallskip}
Cluster & Publications & Conductance & Total ASJC Labels & Unique ASJC Labels\\
\noalign{\smallskip}\hline\noalign{\smallskip}
0 & 111,294 & 0.12 & 265,664 & 142 \\ 
1 & 117,057 & 0.14 & 246,960 & 166 \\ 
2 & 116,251 & 0.11 & 280,602 & 165 \\ 
3 & 353,366 & 0.15 & 881,693 & 200 \\ 
4 & 145,081 & 0.09 & 349,020 & 154 \\ 
5 & 92,097 & 0.17 & 248,168 & 186 \\ 
6 & 71,865 & 0.15 & 199,681 & 163 \\ 
7 & 179,927 & 0.18 & 465,181 & 202 \\ 
8 & 302,656 & 0.18 & 760,117 & 214 \\ 
9 & 69,520 & 0.19 & 197,031 & 174 \\ 
10 & 42,462 & 0.10 & 102,838 & 141 \\ 
11 & 448,030 & 0.25 & 1,229,061 & 224 \\ 
12 & 70,738 & 0.21 & 216,546 & 179 \\ 
13 & 105,232 & 0.17 & 289,318 & 187 \\ 
14 & 199,176 & 0.15 & 551,657 & 208 \\ 
15 & 64,384 & 0.17 & 195,679 & 167 \\ 
16 & 89,340 & 0.11 & 240,157 & 158 \\ 
17 & 50,817 & 0.14 & 181,531 & 179 \\ 
18 & 43,113 & 0.20 & 108,518 & 177 \\ 
19 & 12,615 & 0.80 & 36,583 & 229 \\ \noalign{\smallskip}\hline
\end{tabular}
\end{table}
\newpage

%\begin{acknowledgements}
%If you'd like to thank anyone, place your comments here
%and remove the percent signs.
%\end{acknowledgements}


% Authors must disclose all relationships or interests that 
% could have direct or potential influence or impart bias on 
% the work: 
%
% \section*{Conflict of interest}
%
% The authors declare that they have no conflict of interest.


% BibTeX users please use one of
%\bibliographystyle{spbasic}      % basic style, author-year citations
\bibliographystyle{spmpsci}      % mathematics and physical sciences
%\bibliographystyle{spphys}       % APS-like style for physics
\bibliography{comp}   % name your BibTeX data base

% Non-BibTeX users please use

\end{document}
% end of file template.tex



\subsection{Subsection title}
\label{sec:2}
as required. Don't forget to give each section
and subsection a unique label.
\paragraph{Paragraph headings} Use paragraph headings as needed.
\begin{equation}
a^2+b^2=c^2
\end{equation}

% For one-column wide figures use
\begin{figure}
% Use the relevant command to insert your figure file.
% For example, with the graphicx package use
  \includegraphics{example.eps}
% figure caption is below the figure
\caption{Please write your figure caption here}
\label{fig:1}       % Give a unique label
\end{figure}
%
% For two-column wide figures use
\begin{figure*}
% Use the relevant command to insert your figure file.
% For example, with the graphicx package use
  \includegraphics[width=0.75\textwidth]{example.eps}
% figure caption is below the figure
\caption{Please write your figure caption here}
\label{fig:2}       % Give a unique label
\end{figure*}

\begin{figure}[ht]
\centering
% Use the relevant command to insert your figure file.
% For example, with the graphicx package use
  \includegraphics[scale=0.25]{vlc.pdf}
% figure caption is below the figure
\caption{Schematic representation of variable clustering protocol modified from Small and Sweeney (1985) \cite{small_clustering_1985}. Three parameters are specified: (i) a threshold or starting level based on a quantile of normalized co-citation frequency, (ii) a level increment, and  (iii) a maximum cluster size. Input data is a set of co-cited publications with edge-weight defined by normalized co-citation frequencies. Green clusters are within the max cluster size. At the initial threshold, $t1$, a single cluster below the maximum cluster size, $mcs$ (green), along with one large cluster above it (red) are generated. As the threshold is incremented to $t2$, additional clusters of acceptable size is generated. The cascade continues to completion, which is defined by all clusters being of size less than or equal to the $mcs$. In this schematic, five rounds are adequate for the process to run to completion.}
\label{fig:vlc_process}       % Give a unique label
\end{figure}



\begin{figure}[ht]
\centering
% Use the relevant command to insert your figure file.
% For example, with the graphicx package use
 \includegraphics[scale=0.5]{cocit_venn2.pdf}
% figure caption is below the figure
\caption{High-level disciplinary composition of publications in $\sim$2,000 clusters generated by variable-level clustering  of co-citation data.
%Variable Level Clustering. 
84,591 nodes from were matched to the 4 subject areas, 
27 major subject areas, and 334 minor subject areas in the Scopus ASJC classification. Since 82.5\% of these nodes were assigned more to than one minor subject category, 
we assigned fractional counts for the four subject areas that summed to 1 for each article. The figure shows a four way Venn Diagram~\cite{chen_venn_2011} illustrating the 
relative frequencies of publications from the Physical Sciences, Social Sciences, Life Sciences, and Health Sciences.}
\label{fig:cocitr_venn2}       % Give a unique label
\end{figure}
\newpage

\begin{figure}[ht]
\centering
% Use the relevant command to insert your figure file.
% For example, with the graphicx package use
  \includegraphics[scale=0.5]{merge_select.pdf}
% figure caption is below the figure
\caption{Clustering produced using co-citation data; the y-axis is the cluster size (given logarithmically), and the x-axis is the number of iterations of the agglomerative clustering technique. Note that iterating more than 600 times produces a very large cluster.  }
\label{fig:merge_select}       % Give a unique label
\end{figure}

%
% For tables use

