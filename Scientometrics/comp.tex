%%%%%%%%%%%%%%%%%%%%%%% file template.tex %%%%%%%%%%%%%%%%%%%%%%%%%
%
% This is a general template file for the LaTeX package SVJour3
% for Springer journals.          Springer Heidelberg 2010/09/16
%
% Copy it to a new file with a new name and use it as the basis
% for your article. Delete % signs as needed.
%
% This template includes a few options for different layouts and
% content for various journals. Please consult a previous issue of
% your journal as needed.
%
%%%%%%%%%%%%%%%%%%%%%%%%%%%%%%%%%%%%%%%%%%%%%%%%%%%%%%%%%%%%%%%%%%%
%
% First comes an example EPS file -- just ignore it and
% proceed on the \documentclass line
% your LaTeX will extract the file if required
\begin{filecontents*}{example.eps}
%!PS-Adobe-3.0 EPSF-3.0
%%BoundingBox: 19 19 221 221
%%CreationDate: Mon Sep 29 1997
%%Creator: programmed by hand (JK)
%%EndComments
gsave
newpath
  20 20 moveto
  20 220 lineto
  220 220 lineto
  220 20 lineto
closepath
2 setlinewidth
gsave
  .4 setgray fill
grestore
stroke
grestore
\end{filecontents*}
%
\RequirePackage{fix-cm}
%
%\documentclass{svjour3}                     % onecolumn (standard format)
%\documentclass[smallcondensed]{svjour3}     % onecolumn (ditto)
\documentclass[smallextended]{svjour3}       % onecolumn (second format)
%\documentclass[twocolumn]{svjour3}          % twocolumn
%
\smartqed  % flush right qed marks, e.g. at end of proof
%
\usepackage{graphicx}
%
% \usepackage{mathptmx}      % use Times fonts if available on your TeX system
%
% insert here the call for the packages your document requires
%\usepackage{latexsym}
% etc.
%
% please place your own definitions here and don't use \def but
% \newcommand{}{}
%
% Insert the name of "your journal" with
% \journalname{myjournal}
%
\begin{document}

\title{An Appreciation of Computer Science%\thanks{Grants or other notes
%about the article that should go on the front page should be
%placed here. General acknowledgments should be placed at the end of the article.}
}
\subtitle{Article Level Clusters vs Journals\\ If so, write it here}

%\titlerunning{Short form of title}        % if too long for running head

\author{Sitaram Devarakonda         \and
        George Chacko %etc.
}

%\authorrunning{Short form of author list} % if too long for running head

\institute{Sitaram Devarakonda \at
              Netelabs, NET ESolutions Corporation, McLean, VA \\
              %Tel.: +123-45-678910\\
              %Fax: +123-45-678910\\
              \email{sitaram@nete.com}           %  \\
%             \emph{Present address:} of F. Author  %  if needed
           \and
           George Chacko \at
              Netelabs, NET ESolutions Corporation, McLean, VA \\
              \email{netelabs@nete.com}
}

\date{Received: date / Accepted: date}
% The correct dates will be entered by the editor


\maketitle

\begin{abstract}
Computer science and its applications has experienced dramatic growth and diversification over the last twenty years. Toward understanding the nature of these changes, we analyze a cohort of the computer science literature using a historiographic approach that relies on citations. To escape the imprecise definitions encoded in journal-level groupings, we build article-level clusters. Mention structure and information flow. We compare two different clustering approaches and reconcile them with subject categories in the Scopus database.

\keywords{Bibliometrics \and Clustering \and Knowledge Evolution \and Research Evaluation}
% \PACS{PACS code1 \and PACS code2 \and more}
\subclass{01A85 \and 01A90} %\and more}
\end{abstract}

\section{Introduction}
\label{intro}

Cite Alan Porter on expanding universe versus myopic focus. Cite Traag, Waltman, Lariviere papers. Peter Sjögårde and Per Ahlgren, Loet Leydesdorff, Caroline S. Wagner and Lutz Bornmann, Discontinuities in citation relations among journals: self-organized criticality as a model of scientific revolutions and change, Scientometrics, 10.1007/s11192-018-2734-6, 116, 1, (623-644), (2018). Antonio Perianes-Rodriguez and Javier Ruiz-Castillo, A comparison of the Web of Science and publication-level classification systems of science, Journal of Informetrics, 10.1016/j.joi.2016.10.007, 11, 1, (32-45), (2017). Qi Wang and Ludo Waltman, Large-scale analysis of the accuracy of the journal classification systems of Web of Science and Scopus, Journal of Informetrics, 10.1016/j.joi.2016.02.003, 10, 2, (347-364), (2016). Henry Small, Kevin W. Boyack and Richard Klavans, Identifying emerging topics in science and technology, 
Crossref
\section{Materials and Methods}
\label{sec:1}
Text with citations \cite{RefB} and \cite{RefJ}.

\section{Results}
\label{sec:2}

Overall strategy:

\begin{itemize}
\item Spectral clustering
\begin{enumerate}
\item Subset Scopus by COMP
\item Match COMP to dblp at high stringency -> reduced number of scps
 \item Harvest cited refefences  from Scopus
 \item Cluster articles by direct citation using Graclus to a number that meets simple optimality criteria and does not impose cognitive challenges
 \item Reconcile cluster contents to journal classification for the purpose of referencing
 \item Infer inter-cluster relationships using direct citations, co-citations, and bibliographic coupling
\end{enumerate}
\item Co-citation based clustering
\begin{enumerate}
\item Stratify by year after subsetting by COMP and restricting to article and cp only
\item Apply fractional citation counting and identify highly-cited papers
\item Calculate normalized co-citations
\item Cluster by variable clustering method of Small
\item Evaluate major co-cited pairs
\end{enumerate}
\end{itemize}


Text with citations \cite{RefB} and \cite{RefJ}.

\subsection{Subsection title}
\label{sec:2}
as required. Don't forget to give each section
and subsection a unique label (see Sect.~\ref{sec:1}).
\paragraph{Paragraph headings} Use paragraph headings as needed.
\begin{equation}
a^2+b^2=c^2
\end{equation}

% For one-column wide figures use
\begin{figure}
% Use the relevant command to insert your figure file.
% For example, with the graphicx package use
  \includegraphics{example.eps}
% figure caption is below the figure
\caption{Please write your figure caption here}
\label{fig:1}       % Give a unique label
\end{figure}
%
% For two-column wide figures use
\begin{figure*}
% Use the relevant command to insert your figure file.
% For example, with the graphicx package use
  \includegraphics[width=0.75\textwidth]{example.eps}
% figure caption is below the figure
\caption{Please write your figure caption here}
\label{fig:2}       % Give a unique label
\end{figure*}
%
% For tables use
\begin{table}
% table caption is above the table
\caption{Please write your table caption here}
\label{tab:1}       % Give a unique label
% For LaTeX tables use
\begin{tabular}{lll}
\hline\noalign{\smallskip}
first & second & third  \\
\noalign{\smallskip}\hline\noalign{\smallskip}
number & number & number \\
number & number & number \\
\noalign{\smallskip}\hline
\end{tabular}
\end{table}


%\begin{acknowledgements}
%If you'd like to thank anyone, place your comments here
%and remove the percent signs.
%\end{acknowledgements}


% Authors must disclose all relationships or interests that 
% could have direct or potential influence or impart bias on 
% the work: 
%
% \section*{Conflict of interest}
%
% The authors declare that they have no conflict of interest.


% BibTeX users please use one of
%\bibliographystyle{spbasic}      % basic style, author-year citations
%\bibliographystyle{spmpsci}      % mathematics and physical sciences
%\bibliographystyle{spphys}       % APS-like style for physics
%\bibliography{}   % name your BibTeX data base

% Non-BibTeX users please use
\begin{thebibliography}{}
%
% and use \bibitem to create references. Consult the Instructions
% for authors for reference list style.
%
\bibitem{RefJ}
% Format for Journal Reference
Author, Article title, Journal, Volume, page numbers (year)
% Format for books
\bibitem{RefB}
Author, Book title, page numbers. Publisher, place (year)
% etc
\end{thebibliography}

\end{document}
% end of file template.tex

