\documentclass[11pt, oneside]{article}   	% use "amsart" instead of "article" for AMSLaTeX format
\usepackage{geometry}                		% See geometry.pdf to learn the layout options. There are lots.
\geometry{letterpaper}                   		% ... or a4paper or a5paper or ... 
%\geometry{landscape}                		% Activate for rotated page geometry
%\usepackage[parfill]{parskip}    		% Activate to begin paragraphs with an empty line rather than an indent
\usepackage{graphicx}				% Use pdf, png, jpg, or eps§ with pdflatex; use eps in DVI mode
								% TeX will automatically convert eps --> pdf in pdflatex		
\usepackage{amssymb}
%SetFonts

\title{SCIM-D-19-01397: Response to Review of R1}
\author{George Chacko}
%\date{}							% Activate to display a given date or no date

\begin{document}
\maketitle
%\section{}
%\subsection{}

\vspace{30 mm}
\noindent Dear Dr. Gl{\"a}nzel:\\

Thank you very much for your message of May 11, 2020 in which it was communicated that our manuscript could be considered for publication `after you have carried out the corrections 
suggested by reviewer(s)'. Accordingly, I am (also on behalf of my co-authors) resubmitting a revised manuscript. I have addressed the comments  of the reviewers below. For the second time, 
we thank the reviewers for their constructive critique and timely reviews. If our views diverge from those of the reviewers, we do so respectfully.

\vspace{8 mm}

\noindent Sincerely

\vspace{8 mm}

\noindent George Chacko, PhD

\newpage

\section*{Reviewer 1} 1. By visibility of the images Reviewer means that Fig. 3 (especially a \& b) labels on x and y axis are of very small font. 2. After the revision, paper is better understandable then before. Also, Since authors have 
used DBLP and Scopus, Reviewer would like authors to add  some discussion in the introduction about other two widely used bibliometric sources i.e google scholar and web of science and mention pros and cons of why they have used 
the said two and why not these two. Authors may take hint of such discussion from papers like: Zavadskas, E. K., Skibniewski, M. J., \& Antucheviciene, J. (2014). Performance analysis of Civil Engineering Journals based on the Web of Science® 
database. Archives of Civil and Mechanical Engineering, 14(4), 519-527. Shukla, A. K., Janmaijaya, M., Abraham, A., \& Muhuri, P. K. (2019). Engineering applications of artificial intelligence: A bibliometric analysis of 30 years (1988-2018). 
Engineering Applications of Artificial Intelligence, 85, 517-532.ienc

\subsection*{Response}
\begin{enumerate}
\item Fig 3 has been revised to increase the size of x-axis and y-axis labels and scaled by placing it in landscape mode. Thank you for clarifying our question.
\item We have carefully considered the suggestion to expand the introduction to discuss Google Scholar and Web of Science as alternatives to Scopus. This suggestion is appreciated. 
However, we believe that an expanded discussion of the perceived relative merits of these commercial databases is distracting since the reasons advanced in the literature to recommend one 
over the other are contextual, disciplinary, and often subjective. We have added a sentence, however, mentioning Scopus and Web of Science and citing Archambault 2009 (below).

Our use of DBLP is justified in the Introduction as a `reference bibliography for computer science'. We also mention the limitations of ASJC and other journal based classifications to identify 
disciplines, hence our reliance on article level clusters. In the course of responding to this point. we found a conference paper by Pham and Klamma, that also identifies DBLP as important 
to the field of computer science, and we have included it in our bibliography.

In terms of Scopus, the choice was simple. Our implementation of Scopus is full-featured (all ~90 million records and associated metadata are resident on our servers). Thus, we do not need to scrape 
data or use a throttled API to access information. We also chose Scopus over Web of Science because of its its well designed XML structure. We previously used Web of Science (also as resident on 
our servers) and our experience is that the two are comparable in terms of coverage but clearly not identical. This is consistent with Eric Archambault's observations.  Lastly, our budget does not permit us to afford two subscriptions. 

Google Scholar is a non-starter as far as we are concerned. It suffers from false positives, duplicates, includes grey literature, and does not allow large scale analysis. As mentioned by Martin-Martin et al. (2019),
`Google Scholar, Web of Science, and Scopus: Which is best for me?' on the London School of Economics website, \emph{``The question, as our colleague Professor Harzing put it, is whether 
we are ready to accept a trade-off: going beyond the comfortable and orderly borders of curated databases in exchange for more diverse coverage.''} We do not consider this trade-off useful for 
our purposes. Accordingly, we find value in the orderly borders of the Scopus and DBLP curated databases over the use of Google Scholar for this study.

We examined the references suggested by the reviewer; Zavadskas et al. (2014) and Shukla et al. (2019). The text within is not inconsistent with the points we make in our first paragraph but did not 
provide additional insight that would persuade us to expand the Introduction further. For example, Shukla et al. state that.``WoS is the first multidisciplinary bibliographic index of journal publications designed. 
It is considered a standard data source for bibliometrics. Other databases such as Scopus by Elsevier are also used by bibliometricians." and ``A comparison with other indexing platforms such as Scopus, 
Google Scholar can be a good future scope of this study.''

To examine the issue raised by the reviewer, we read some comprehensive comparisons of these databases but did not find compelling reasons to further expand the Introduction. We have listed a few 
of these sources below. Archambault's paper is relevant (or at least consistent with) our comment that the choice of either Web of Science or Scopus is contextual and can be subjective. 
MacRoberts and MacRoberts assert that all bibliographic databases have limitations and we agree. 

\begin{itemize}
\item Yang and Meho (2007) Citation Analysis: A Comparison of Google Scholar, Scopus, and Web of Science JASIST 10.1002/meet.14504301185
\item Falagas et al. (2007) Comparison of PubMed, Scopus, Web of Science, and Google Scholar: Strengths and Weaknesses FASEB J. 10.1096/fj.07-9492LSF
\item Archambault et al. (2009)  Comparing bibliometric statistics obtained from the Web of Science and Scopus. JASIST  10.1002/asi.21062
\item Harzing and Alakangas (2016) Google Scholar, Scopus and the Web of Science: a longitudinal and cross-disciplinary comparison Scientometrics 10.1007/s11192-015-1798-9
\item MacRoberts and MacRoberts (2018) JASIST 10.1002/asi.23970
\end{itemize}
\end{enumerate}


\section*{Reviewer 2} 

This manuscript overviews the fields of computer science by the means of clustering for networks of direct citation or co-citation. Comparing with the previous version, the manuscript 
becomes easy for readers to understand the content along the purpose. However, there are some mis-spelligs or mis-formattings.
- Please search "co-citation. .", and delete one of these periods.
- Please search "by takimg advantage of modern bibliographic resources", and change "takimg" to "taking".
- Please search "discuss our findings, Overall", and change from comma to period.
- Please search "y:axis", and change to "y-axis".

The formats of second-level headings are mixed like "Co-Citation." and "Scopus data", whose difference is to finish with a period or without it. Please unify the format, and the reviewer recommends 
the italic (and bold) "Co-Citation:", refering to\\ http://journals.ieeeauthorcenter.ieee.org/wp-content/uploads/sites/7/IEEE-Editorial-Style-Manual.pdf.
- "DBLP dataA" must be "DBLP data A".
- I guess that "Merging and graph construction" is also one of the second-level headings.
I have judged that these are proofreading tasks, but the number of these is large. Thus, please check your manuscript to repair careless mistakes by yourself. I expect I cannot find all.
As a result, I recommend minor revision close to proofreading-level. About the content, there is no problem

\subsection*{Response}

We thank the reviewer for these comments. We have corrected all the typographic errors noted and made further corrections at the proofreading level. Some of them seem to have been generated by 
the \emph{changes} package juxtaposing sections of text when run in clean mode. We have constructed a clean copy.

    
\end{document}  