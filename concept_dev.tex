\documentclass[11pt, oneside]{article}   	% use "amsart" instead of "article" for AMSLaTeX format
\usepackage{geometry}                		% See geometry.pdf to learn the layout options. There are lots.
\geometry{letterpaper}                   		% ... or a4paper or a5paper or ... 
%\geometry{landscape}                		% Activate for rotated page geometry
%\usepackage[parfill]{parskip}    		% Activate to begin paragraphs with an empty line rather than an indent
\usepackage{graphicx}				% Use pdf, png, jpg, or eps§ with pdflatex; use eps in DVI mode
								% TeX will automatically convert eps --> pdf in pdflatex		
\usepackage{amssymb}

%SetFonts

%SetFonts


\title{Concept Dev for COMP study}
\author{Sitaram Devarakonda \& George Chacko}
%\date{}							% Activate to display a given date or no date

\begin{document}
\maketitle
\section{Motivation} Making discovery in a manner that would leverage existing resources and skills with focus on  the computer science field, which has experienced vigorous growth in the last twenty years. Citation analysis is a particular focus. 

\section{Data and Methods} The following sections describe the datasets and initial data exploration 
\subsection{Year slices} Year slices were developed for all publications from 1996-2018. All publications of language 'English' and with at-least 2 references and complete publication data in Scopus are selected. Publications type include (article, conference paper, dissertation etc) and belong to subject area 'COMP'. 
%\vspace{-7mm}
% latex table generated in R 3.5.1 by xtable 1.8-4 package
% Thu Aug 15 14:13:47 2019
\begin{table}[ht]
\caption{Summary of initial Scopus Analytical data set. The number of unique publications, unique references, total references . } 
\label{tab:summary_data}
\vspace{1mm}
\centering
\scalebox{0.7}{
\begin{tabular}{|r rrr r|}
  \hline
 & Year & Unique Publications & Unique References & Total References \\ 
  \hline
1 & 1996 & 30783 & 185816 & 320533 \\ 
  2 & 1997 & 37284 & 230475 & 389779 \\ 
  3 & 1998 & 45198 & 254160 & 445703 \\ 
  4 & 1999 & 40776 & 241013 & 420839 \\ 
  5 & 2000 & 45310 & 266764 & 471079 \\ 
  6 & 2001 & 65168 & 357472 & 651196 \\ 
  7 & 2002 & 75583 & 410290 & 761714 \\ 
  8 & 2003 & 88701 & 458459 & 904755 \\ 
  9 & 2004 & 88984 & 506063 & 980388 \\ 
  10 & 2005 & 110741 & 609294 & 1247543 \\ 
  11 & 2006 & 140729 & 731622 & 1579690 \\ 
  12 & 2007 & 171357 & 845569 & 1879419 \\ 
  13 & 2008 & 205852 & 1019330 & 2281452 \\ 
  14 & 2009 & 239905 & 1239406 & 2841795 \\ 
  15 & 2010 & 255254 & 1391036 & 3214371 \\ 
  16 & 2011 & 255633 & 1534865 & 3464829 \\ 
  17 & 2012 & 264011 & 1700862 & 3845051 \\ 
  18 & 2013 & 261442 & 1866133 & 4199482 \\ 
  19 & 2014 & 272711 & 2004406 & 4581855 \\ 
  20 & 2015 & 309446 & 2220732 & 5202960 \\ 
  21 & 2016 & 316430 & 2422531 & 5630283 \\ 
  22 & 2017 & 308102 & 2622125 & 5918274 \\ 
  23 & 2018 & 340221 & 3012740 & 6960056 \\ 
   \hline
\end{tabular}}
\vspace{-1mm}
\end{table}

\subsection{Disciplinary Composition} One way to identify the growth and diversity of a field is by considering the count and subject areas of scientific publications in it and how these change over time. In our study of the evolution of the field of computer science, we initially used the All Science Journal Classification (ASJC) codes assigned by Scopus to journals to crudely estimate publications and unique references from 1996-2018. Accordingly, we searched for articles or conference proceedings labelled computer science (COMP), one of the 27 major subject areas in the Scopus classification system. ICOMP itself comprises 13 minor subject areas such as Hardware and Architecture and Artificial Intelligence. A limitation of the ASJC system is that every publication inherits its journal or conference subject area(s), which is problematic for a field that has had broad influence and interactions on other fields. Further, multidisciplinary journals such as Nature, Science, Proceedings of the National Academy of Sciences are classified under General and computer science articles published in those journals would be overlooked. Thus, we followed an initial subsetting of Scopus using COMP and GENERAL with an article level clustering approach. :

\begin{description}
\item[$\bullet$ ] NSF follows a single subject area for journals which can be used for publication level classification.
\item[$\bullet$ ] Ludo Waltman methodology for publication-level classification (uses direct citation)
\item[$\bullet$ ] Kevin Boyack methodology 
\item[$\bullet$ ] Using Scopus FPE to generate publication level classification
\item[$\bullet$ ] Obtain top conferences/journals in COMP field and analyze their citation patterns. (Assuming they contain articles related to single discipline)
\end{description}

\subsection{Possible next steps(based on discussion)}  Analyze the citation patterns of top authors in the field and how it evolved over time. Selecting authors with at least 30 papers published might be a good place to start. Another experiment might be to approach the diversity of computer science from other disciplines. i.e studying the citation patterns of other disciplines and how computer science has been more inclusive over time.
\end{document}  