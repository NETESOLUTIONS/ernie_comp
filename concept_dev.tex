\documentclass[11pt, oneside]{article}   	% use "amsart" instead of "article" for AMSLaTeX format
\usepackage{geometry}                		% See geometry.pdf to learn the layout options. There are lots.
\geometry{letterpaper}                   		% ... or a4paper or a5paper or ... 
%\geometry{landscape}                		% Activate for rotated page geometry
%\usepackage[parfill]{parskip}    		% Activate to begin paragraphs with an empty line rather than an indent
\usepackage{graphicx}				% Use pdf, png, jpg, or eps§ with pdflatex; use eps in DVI mode
								% TeX will automatically convert eps --> pdf in pdflatex		
\usepackage{amssymb}

%SetFonts

%SetFonts


\title{Concept Dev for COMP study}
\author{Sitaram Devarakonda \& George Chacko}
%\date{}							% Activate to display a given date or no date

\begin{document}
\maketitle
\section{Motivation} Making discovery in a manner that would leverage existing resources and skills with focus on  the computer science field, which has experienced vigorous growth in the last twenty years. Citation analysis is a particular focus. 

\section{Data and Methods} The following sections describe the datasets and initial data exploration 
\subsection{Year slices} Year slices were developed for all publications from 1996-2018. All publications of language 'English' and with at-least 2 references and complete publication data in Scopus are selected. Publications type include (article, conference paper, dissertation etc) and belong to subject area 'COMP'. 
%\vspace{-7mm}
\begin{table}[ht]
\caption{Summary of base Scopus Analytical data set. The number of unique publications, unique references, total references . } 
\label{tab:summary_data}
\centering
\scalebox{0.7}{
\begin{tabular}{|r  ccc |}
  \hline
Year & Unique Publications & Unique References  & Total References \\ 
  \hline
1996 & 30,740 & 184,617 & 318,700 \\ 
1997 &37,166 & 228,464 & 386,627 \\
1998 & 44,764 & 250,491 & 438,908 \\
1999 & 40,434 & 237,175 & 413,681 \\
2000 & 44,636 & 261,393 & 460,310 \\
2001 & 62,000 & 343,937 & 619,330 \\
2002 & 71,645 & 394,882 &722,278 \\
2003 & 83,676 & 438,498 & 849,950 \\
2004 & 82,914 & 482,475 & 915,997 \\
2005 & 107,905 & 591,784 & 1,193,198 \\
2006 & 139,379 & 713,924 & 1,531,101 \\
2007 & 170,396 & 829,444 & 1,830,944 \\
2008 & 204,084 & 1,000,790 & 2,226,980 \\
2009 & 239,113 & 1,222,312 & 2,790,194 \\
2010 & 254,763 & 1,374,471 & 3,164,912 \\
2011 & 255,144 & 1,519,064 & 3,419,719 \\
2012 & 170,364 & 1,374,757 & 2,785,900 \\
2013 & 195,008 & 1,606,955 & 3,353,390 \\
2014 & 264,968 & 1,863,488 & 4,260,613 \\
2015 & 297,534 & 2,037,838 & 4,762,036 \\
2016 & 305,372 & 2,217,138 & 5,140,149 \\
2017 & 291,481 & 2,349,259 & 5,262,725 \\
2018 & 316,495 & 2,623,449 & 6,009,386 \\
 \hline
\end{tabular}}
\vspace{-1mm}%Put here to reduce too much white space after your table 
\end{table}

\subsection{Disciplinary Composition} One way to identify the growth and diversity of a field might be the subject areas of publications and how it changes over time .So We  calculated the disciplinary composition for all the publications and unique references from 1996-2018. These are calculated based on ASJC codes which are assigned by Scopus at journal level, i.e every publication in a journal or conference will have the same subject area. Journals or conferences can have multiple subject areas. Multidisciplinary journals such as Nature, Science, Proceedings of the National Academy of Sciences are classified as General which result in the loss of computer science articles published in those journals. Also there is a lot of ambiguity in journals/conferences with multiple ASJC codes. Following possibilities can be explored to overcome ambiguity:

\begin{description}
\item[$\bullet$ ] NSF follows a single subject area for journals which can be used for publication level classification.
\item[$\bullet$ ] Ludo Waltman methodology for publication-level classification (uses direct citation)
\item[$\bullet$ ] Kevin Boyack methodology 
\item[$\bullet$ ] Using Scopus FPE to generate publication level classification
\item[$\bullet$ ] Obtain top conferences/journals in COMP field and analyze their citation patterns. (Assuming they contain articles related to single discipline)
\end{description}

\subsection{Possible next steps(based on discussion)}  Analyze the citation patterns of top authors in the field and how it evolved over time. Selecting authors with at least 30 papers published might be a good place to start. Another experiment might be to approach the diversity of computer science from other disciplines. i.e studying the citation patterns of other disciplines and how computer science has been more inclusive over time.
\end{document}  